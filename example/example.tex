\documentclass[a4paper,14pt]{extarticle}

% кодировка
\usepackage[utf8]{inputenc}
\usepackage[T2A]{fontenc}

% поля
\usepackage[left=30mm,right=15mm,top=20mm,bottom=20mm]{geometry}

% переносы слов
\usepackage[english,russian]{babel}

% Таймс шрифт
\usepackage{tempora}

% полуторный межстрочный интервал
\usepackage{setspace}
\onehalfspacing

% отступ первой строки
\usepackage{indentfirst}
\setlength{\parindent}{1.25cm}

% структурный элемент
\newcommand{\hidedstructel}[1]{
    \clearpage
    \section*{#1}
}

\newcommand{\structel}[1]{
    \hidedstructel{#1}
    \addcontentsline{toc}{section}{#1} 
} 

\usepackage{titlesec}
\titleformat{\section}
    [display]               % форма
    {\filcenter\bfseries}   % формат всего заголовка
    {ПРИЛОЖЕНИЕ \thesection}                      % метка
    {}                      % отступ от метки
    {}                      % код перед телом

% раздел
\newcommand{\sect}[1]{
    \clearpage
    \setcounter{figure}{0}  % сбросить нумерацию внутри раздела
    \setcounter{table}{0}
    \setcounter{listing}{0}
    \subsection{#1}

    \renewcommand{\theparagraph}{\thesubsection.\arabic{paragraph}}
}
\titleformat{\subsection}{\filright\bfseries}{}{}{\thesubsection\hspace{1em}}
\titlespacing*{\subsection}
    {\parindent}    % отступ слева
    {}              % до
    {}              % после
\renewcommand{\thesubsection}{\arabic{subsection}}

\renewcommand{\thesection}{\Asbuk{section}}
\newcommand{\pril}[1]{
    \clearpage
    \section{#1}
}

% подраздел
\usepackage{placeins}
\newcommand{\subsect}[1]{
    \FloatBarrier
    \subsubsection{#1}

    \renewcommand{\theparagraph}{\thesubsubsection.\arabic{paragraph}}
}
\titleformat{\subsubsection}{\filright\bfseries}{}{}{\thesubsubsection\hspace{1em}}
\titlespacing*{\subsubsection}{\parindent}{}{}

% пункт
\newcommand{\parag}{
    \paragraph{}
}
\titleformat{\paragraph}[runin]{}{\theparagraph}{1em}{ }  % для отступа
\titlespacing*{\paragraph}{\parindent}{}{}

% подпункт
\newcommand{\subparag}{
    \subparagraph{}
}
\titleformat{\subparagraph}[runin]{}{\thesubparagraph}{1em}{ }
\titlespacing*{\subparagraph}{\parindent}{}{}

% содержание
\usepackage{etoc}

\setcounter{tocdepth}{3}
\setcounter{secnumdepth}{5}

% перечисления
\usepackage{enumitem}
\setlist{
    topsep=0,                   % отступ сверху и снизу списка
    partopsep=0,                % то же самое
    leftmargin=0,               % отступ слева
    labelsep=0,                 % отступ метки
    align=left,                 % выравнивание метки
    listparindent=\parindent,   % отступ первой строки абзаца
    itemsep=0,                  % отступ между элементами
    parsep=0                    % отступ между абзацами и элементами
}
\setlist[itemize]{
    label=--~,  % в списках тире короткое, в тексте - длинное
    labelwidth=1.2em,
    itemindent=\parindent+\labelwidth
}
\setlist[enumerate]{
    label=\arabic*),
    labelwidth=1.4em,
    itemindent=\parindent+\labelwidth
}
%\makeatletter
%\AddEnumerateCounter{\asbuk}{\russian@alph}{щ}
%\makeatother
\AddEnumerateCounter*{\asbuk}{\c@asbuk}{7}
\newlist{asblist}{enumerate}{2}
\setlist[asblist]{
    label=\asbuk*),
    labelwidth=1.4em,
    itemindent=\parindent+\labelwidth
}

% подписи
\usepackage[singlelinecheck=false]{caption}
\DeclareCaptionLabelSeparator{gost}{~---~}

% иллюстрация
\newcommand{\fig}[3][1]{
    \begin{figure}[h]
        \centering
        \includegraphics[width=#1\textwidth]{#2}
        \caption{#3}\label{#2}
    \end{figure}
}
\renewcommand{\thefigure}{\thesubsection.\arabic{figure}}
\DeclareCaptionLabelFormat{gostfigure}{Рисунок #2}
\captionsetup[figure]{justification=centering, labelformat=gostfigure, position=bottom}
% font=singlespacing по умолчанию



% листинг
\usepackage[newfloat, cache=false]{minted}

\newcommand{\lst}[2]{
    \begin{listing}[h]
        \centering
        \caption{#2}\label{#1}
        \begin{minipage}[t]{.8\textwidth}
            \inputminted[
                fontsize=\small,
                frame=single,
                breaklines,
                linenos
            ]{text}{#1}
        \end{minipage}
    \end{listing}
}
\renewcommand{\theFancyVerbLine}{\rmfamily{\small \oldstylenums{\arabic{FancyVerbLine}}}} 
\DeclareCaptionLabelFormat{custlisting}{Листинг #2}
\captionsetup[listing]{justification=raggedright, labelformat=custlisting, position=top}
%skip=-6pt
\renewcommand{\thelisting}{\thesubsection.\arabic{listing}}

\newenvironment{codepiece}[2]
{
    \VerbatimEnvironment
    \begin{listing}[h]
        \centering
        \caption{#2}\label{lst:#1}
        \begin{minipage}[t]{.8\textwidth}
            \begin{minted}[
                fontsize=\small,
                frame=single,
                breaklines,
                linenos
            ]{text}
}{
            \end{minted}
        \end{minipage}
    \end{listing}
}

% код в таблице
\newenvironment{tabcode}[1]
{
    \VerbatimEnvironment
    \begin{minipage}[t]{#1\textwidth}
    \begin{minted}[fontsize=\small, breaklines]{text}
}{
    \end{minted}
    \end{minipage}
}

\usepackage{multirow}
\usepackage{longtable}
\usepackage{tabularx}
\renewcommand{\thetable}{\thesubsection.\arabic{table}}

% таблица
\newenvironment{tbl}[3]
{
    \begin{table}[h]
        \small
        \centering
        \caption{#2}\label{tbl:#1}
        \begin{tabular}{|#3|}
            \hline
}{
            \hline
        \end{tabular}
    \end{table}
}

\newenvironment{longtbl}[3]
{
    \small
    \begin{longtable}{|#3|}
        \caption{#2}\label{tbl:#1}\\
        \hline
}{
        \hline
    \end{longtable}
}

%\newcommand{\mr}[2]{\multirow[t]{#1}{=}{#2}}



% математика
\usepackage{amsmath}

% графики
\usepackage{tikz, pgfplots}
\pgfplotsset{compat=newest}

% таблицы
\usepackage{array}

\newcolumntype{M}[1]{>{\centering\arraybackslash}m{#1}}
\newcolumntype{N}[1]{>{\raggedright\arraybackslash}p{#1}}

\usepackage{xparse}
\NewExpandableDocumentCommand\thead{t< t> O{1} m m}{% 
    \IfBooleanTF{#1}{%
        \IfBooleanTF{#2}{%
            \multicolumn{#3}{|M{#4}|}{#5}%
        }{%
            \multicolumn{#3}{|M{#4}}{#5}%
        }
    }{%
        \IfBooleanTF{#2}{%
            \multicolumn{#3}{M{#4}|}{#5}%
        }{%
            \multicolumn{#3}{M{#4}}{#5}%
        }%
    }%
}




\usepackage{adjustbox}
\usepackage{float}


\DeclareCaptionLabelFormat{gosttable}{Таблица #2}

\captionsetup[table]{justification=raggedright, labelformat=gosttable, position=top}
\captionsetup{labelsep=gost}


% для отладки
%\usepackage{showframe}
%\renewcommand\ShowFrameLinethickness{0.25pt}
%\renewcommand*\ShowFrameColor{\color{red}}
% картинки
\usepackage{graphicx}




\begin{document}

\newcommand{\signplace}{\underline{\hspace{40mm}}}
\newcommand{\dateblank}{%
    <<\underline{\hspace{10mm}}>> \underline{\hspace{30mm}} 2020 г.%
}
\newlength{\twointerv}\setlength{\twointerv}{28.34pt}

\begin{titlepage}
    \singlespacing
    \setlength{\parindent}{0pt}
    \begin{center}
        Министерство науки и высшего образования Российской Федерации\\
        Фереральное государственное бюджетное образовательное учреждение
высшего образования\\
        ЮГОРСКИЙ ГОСУДАРСТВЕННЫЙ УНИВЕРСИТЕТ\\
        (ЮГУ)
    \end{center}

    \vspace{\twointerv}

    УДК: 004.89

    \vspace{\twointerv}

    \setlength{\tabcolsep}{0pt}
    \begin{tabular}{N{0.5\textwidth}N{80mm}}
        СОГЛАСОВАНО                 & УТВЕРЖДАЮ\\
        Доцент ИЦЭ,                 & Директор ИЦЭ,\\
        к.т.н                       & к.э.н\\
        \signplace{} Л.Н. Толстой   & \signplace{} А.П. Чехов\\
        \dateblank{}                & \dateblank{}
    \end{tabular}

    \vspace{\twointerv}

    \begin{center}
        ПОЯСНИТЕЛЬНАЯ ЗАПИСКА\\
        К ВЫПОЛНЕННОЙ РАБОТЕ\\
        \vspace{\twointerv}
        СТИЛЬ, ОБЕСПЕЧИВАЮЩИЙ СООТВЕТСТВИЕ ДОКУМЕНТОВ, ВЫПОЛНЕННЫХ В СИСТЕМЕ
ВЕРСТКИ LATEX, ГОСТ 7.32-2017\\
        (заключительный)
    \end{center}

    \vfill

    \begin{tabular}{N{70mm}N{80mm}}
        Научный руководитель,\\
        к.т.н & \signplace{} А.С. Пушкин\\
        \vspace{5mm}
        Исполнитель,\\
        студент группы 1162б & \signplace{} И.Р. Панчишин
    \end{tabular}

    \vfill

    \begin{center}
        Ханты-Мансийск 2020
    \end{center}
\end{titlepage}

\setcounter{page}{2}


\hidedstructel{РЕФЕРАТ}

Пояснительная записка \pageref{LastPage} с., \total{figcount} рис.,
\total{tblcount} табл., \total{bibcount} источн., \total{annexcount} прил.

ДЕМОНСТРАЦИЯ, СТИЛЬ, СТАНДАРТ, ПРИМЕР ИСПОЛЬЗОВАНИЯ.

Объектом исследования в данной работе является межгосударственный стандарт под
номером 7.32.

Цель работы --- написать стиль для системы верски \LaTeX, обеспечивающий
соответствие документов стандарту ГОСТ 7.32-2017.

\begingroup

\parindent 0pt
\newlength{\pagewidth}\setlength{\pagewidth}{1.1em}

\newlength{\sectnum}\setlength{\sectnum}{8.4em}
\newlength{\ssectnum}\setlength{\ssectnum}{1em}
\newlength{\sssectnum}\setlength{\sssectnum}{2em}

\newlength{\sssectindent}\setlength{\sssectindent}{2em}

\newcommand*{\entrybody}{%
    \raggedright%
    \etocname\nobreak%
    \leaders\etoctoclineleaders\hfill%
    \rlap{\makebox[\pagewidth][r]{\etocpage}}%
    \vspace{0.56em}% хак для отступа
}

\etocsetstyle{section}
    {}
    {\leavevmode\etocifnumbered{\leftskip \sectnum}{\leftskip 0}}
    {\normalsize\etocifnumbered%
        {\llap{\makebox[\sectnum][l]{ПРИЛОЖЕНИЕ \etocnumber}}%
            \parbox[t][][t]{\textwidth-\sectnum-\pagewidth}{\entrybody}}%
        {\parbox[t][][t]{\textwidth-\pagewidth}{\entrybody}}\par}
    {}
\etocsetstyle{subsection}
    {}
    {\leavevmode\leftskip \ssectnum}
    {\normalsize\llap{\makebox[\ssectnum][l]{\etocnumber}}%
        \parbox[t][][t]{\textwidth-\ssectnum-\pagewidth}{\entrybody}\par}
    {}
\etocsetstyle{subsubsection}
    {}
    {\leavevmode\setlength{\leftskip}{\sssectnum+\sssectindent}\relax}
    {\normalsize\llap{\makebox[\sssectnum][l]{\etocnumber}}%
        \parbox[t][][t]{\textwidth-\sssectnum-\pagewidth-\sssectindent}{\entrybody}\par}
    {}

\etocsettocstyle{\hidedstructel{СОДЕРЖАНИЕ}}{}

\tableofcontents

\endgroup


\structel{ТЕРМИНЫ И ОПРЕДЕЛЕНИЯ}

В настоящей пояснительной записке применяют следующие термины с
соответствующими определениями:

Обозначение

Заголовок

\structel{ВВЕДЕНИЕ ВВЕДЕНИЕ ВВЕДЕНИЕ ВВЕДЕНИЕ ВВЕДЕНИЕ ВВЕДЕНИЕ}

Рассматриваемый стандарт не является строгим: он не описывает все детали
оформления документа. Это позволяет использующим стандарт организациям уточнять
его по своему усмотрению или сочетать с другими стандартами.

Данный документ использует разработанный стиль, поэтому вы имеете возможность
оценить его во время чтения. В документе будут приводиться ссылки на внесенные
изменения.

Используется полуторный межстрочный интервал в ожидании, что объем документа
меньше 500 страниц.

Размер шрифта состовляет 14 пунктов, хотя стандарт допускает и 12 пунктов.

В стандарте рекомендуется использовать семейство шрифтов <<Times New Roman>>,
но оно не является свободным. Поэтому здесь используется семейство со схожим
начертанием.

Полужирный шрифт используется только в заголовках. \emph{Термины можно выделять
курсивом}.

Абзацный отступ равен 1,25 см, левое поле --- 30 мм, правое --- 15 мм, верхнее
и нижнее --- 20 мм.

Номер страницы ставится внизу посередине, титульный лист включается в
нумерацию, но на нем номер не ставится.

\sect{Оформление содержания}

\subsect{Анализ рекомендаций}

Про содержание в стандарте написано в подразделах 5.4 и 6.13.

Каждая запись выравнена по левому краю, номер страницы --- по правому. Номер
страницы соединяется отточием. Подразделы начинаются после абзацного отступа,
равного двум знакам.

Про межстрочный интервал и расстояние между записями ничего не написано, но
указано, что <<запись содержания оформляют как отдельный абзац>>, поэтому
данные параметры те же, что и в основной части документа.

Заголовок раздела при переносе продолжается от уровня своего начала, заголовок
приложения --- <<от уровня записи обозначения этого приложения>>. Рекомендация
по приложению может быть не ясна. Чтобы точно понять, какой отступ у заголовка
приложения на новой строке, можно обратиться к ГОСТ 1.5-2001. По этому госту
оформляются другие госты, и в нем содежится та же рекомендация, с той же
формулировкой. Это значит, что можно посмотреть на отступ, например, в ГОСТ
7.32 и сделать так же.

\subsect{Про пункты в содержании}

В содержание не включаются пункты. По этому поводу в стандарте имеются
противоречия. В пункте 6.2.3 написано, что <<пункты и подпункты могут иметь
только порядковый номер без заголовка>>, но несмотря на это, в 5.4 допускается,
что пункт может иметь заголовок.

\sect{Оформление обозначений}

\parag

Раздел может быть разбит на пункты. Обозначение пунктов состоит из обозначения
включающего уровня и прядкового номера в его пределах, разделенных точкой. Этим
рекомендациям также соответствуют подпункты, разделы и подразделы.

Про расстояние между обозначением и заголовком в стандарте ничего не
сказано. Я определил его длиной одного символа.

\parag

Один пункт не нумеруется, поэтому их два.

\subparag

Пункт может быть разбит на подпункты.

\subparag

Смысла разбивать пункт на подпункты, если подпункт один --- нет.

\sect{Заголовок, который состоит из двух предложений. Они отделяются точной}

Переносы слов в заголовках не допускаются, поэтому он выравнивается по левому
краю.

В стандарте не сказано, как продолжать заголовок на следующей строке, но есть
пример. Отступ сверху и снизу заголовка также не определяется.

\subsect{Подраздел, разбитый на пункты}

\parag

Подраздел также может быть разбит на пункты и подпункты.

\parag

Нумерация пунктов и подпунктов включает нумерацию предыдущего уровня.

Про перечисления в госте написано  в пункте 6.4.6 
Пример сложного перечисления из госта:

\begin{itemize}
    \item в машиностроении:
    \begin{enumerate}
        \item для очистки отливок от формовочной смеси;
        \item для очистки лопат ок т урбин авиационных двигателей;
        \item для холодной штамповки из лист а;
    \end{enumerate}
    \item в ремонте техники:
    \begin{enumerate}
        \item устранение наслоений на внут ренних стенках труб;
        \item очистка каналов и отверстий небольшого диаметра от грязи.
    \end{enumerate}
\end{itemize} 

В стандарте в одном из примеров можно увидеть использование нескольких абзацев
в одном элементе перечисления. Здесь можно так же.

Стандарт не определяет отступ вложенных перечислений, но судя по приведенным примерам в госте,
его нет. Здесь я тоже его убрал.

\begin{asblist}
    \item Допускается использовать лиюбой из трех типов меток элементов,
    \item вне зависимости от расположения перечисления и вложенности (на этот необходимо ссылаться).
\end{asblist}
  
\subsect{Иллюстрации, листинги, таблицы}

Пример иллюстрации, уменьшенной в 2 раза, приведен на рисунке~\ref{example-image-a}.

\fig[0.5]{example-image-a}{Очень длинное название иллюстрации, которое имеет один межстрочный интервал} 

Иллюстрация располагается как можно ближе к ссыке, ссылка указывается обязательно.
Иллюстрации нумеруются в пределах раздела, хотя допускается общая нумерация.
Иллюстрации могут не имень названия, как в случае с рисунком~\ref{example-image-b}.

\fig[0.6]{example-image-b}{} 

Содержимое файла с программным кодом представлено в листинге~\ref{main.c}.

\lst{main.c}{Привет, C!}

В стандарте нет рекомендаций по оформлению листингов. Я сделал листинги похожими на таблицу.

Можно писать код в файле документа. Такой код представлен в листинге~\ref{lst:in-place code}.

Оформление таблиц схоже с оформлением иллюстраций.
Если графы или строки таблицы были пронумерованы,
в первой части, то в последующих можно указывать только их. Заголовки граф и строк указываются
без точки вконце, в единственном числе. Заголовки граф выравниваются по центру, строк --- по
левому краю. Горизонтальные и вертикальные линии \emph{внутри} таблицы допускается не проводить.
Текст в таблице может быть меньше основного текста.

Таблица~\ref{tbl:simple} служит простым примером.

\begin{tbl}{simple}
    {Среднемесячная начисленная заработная плата наёмных работников в
организациях, у индивидуальных предпринимателей и физических лиц}
    {N{5cm}*{5}{M{1cm}}}

    \thead<{5cm}{Субъект} & \thead>[5]{5cm+8\tabcolsep}{Год} \\ 
    & \thead{1cm}{2015} & \thead{1cm}{2016} & \thead{1cm}{2017} &
\thead{1cm}{2018} & \thead>{1cm}{2019} \\\hline

ХМАО & 54815 & 58060 & 60147 & 63150 & 64622 \\\hline
Московская область & 38172 & 38830 & 41921 & 45377 & 46417 \\\hline
г. Москва & 57971 & 59823 & 62186 & 68176 & 72331 \\\hline
Ленинградская область & 29305 & 31462 & 34549 & 38372 & 39280 \\\hline
г. Cанкт-Петербург & 41540 & 42001 & 46516 & 51098 & 54244 \\\hline
Республика Саха (Якутия) & 48955 & 53031 & 55512 & 61474 & 64437 \\\hline
Чукотский авт. округ & 71690 & 79502 & 83969 & 89976 & 97125 \\\hline
Республика Дагестан & 15702 & 17301 & 18633 & 21121 & 21225 \\

\end{tbl}

Большую тарблицу допускается переносить на другую страницу. Перенесенные части
тиблицы имеют специальный заголовок. 

Таблица~\ref{tbl:long} разбита на несколько страниц.

\begin{codepiece}{in-place code}{Привет, LaTeX!}
    \newcommand{\sayhello}{Привет, мир!}
    \sayhello
\end{codepiece}

\begin{longtbl}{long}
    {CD-требования}
    {N{0.4cm}|N{6.5cm}|N{6.5cm}|N{0.4cm}}
        
№ & \thead>{6.5cm}{С-требование} & \thead>{6.5cm}{D-требование} & № \\\hline
\endfirsthead

\caption*{Продолжение таблицы \thetable} \\
\hline
№ & \thead>{6.5cm}{С-требование} & \thead>{6.5cm}{D-требование} & № \\\hline
\endhead

\mr{3}{1} & \mr{3}{Игрок может нарисовать себе флаг}
& Имеется режим редактирования рисунка на флаге & 1 \\\cline{3-4}
&& Приложение предоставляет несколько цветов для рисования & 2 \\\cline{3-4}
&& Если игрок не рисует флаг, то он считается пустым & 2 \\\cline{3-4}
&& Имеется кнопка для очистки флага & 3 \\\hline
\mr{4}{2} & \mr{4}{Во время игры можно отправлять сообщения сопернику}
& Соперник видит сообщение в виде всплываюего текста на противоположном поле & 4 \\\cline{3-4}
&& Пользователь не может ввести больше 100 символов в сообщении & 5 \\\cline{3-4}
&& Сообщение исчезает после 10 секунд & 6 \\\cline{3-4}
&& Сообщение появляется в случайном месте в пределах поля соперника & 7 \\

\end{longtbl}

Значение четвертого члена числовой последовательности Фибоначчи определяется
по формуле~\eqref{eq:fib}.

\begin{multline}\label{eq:fib}
    F(4) = F(3) + F(2) = \\
    = F(2) + F(1) + F(1) + F(0) = F(1) + F(0) + \\
    + F(1) + F(1) + F(0) = 3F(1) + 2F(0) = 3
\end{multline}

где $F(1)$ --- первый член последовательности, равный $1$,\\
$F(0)$ --- нулевой член последовательности, равный $0$.

Документация \cite{bib}.

Хорошая статья \cite{ml}.

\showbib

\annex{Очень интересная схема Очень интересная схема Очень интересная схема}

\annex{Другая схема}

\end{document} 
