\documentclass[a4paper,14pt]{extarticle}

% кодировка
\usepackage[utf8]{inputenc}
\usepackage[T2A]{fontenc}

% поля
\usepackage[left=30mm,right=15mm,top=20mm,bottom=20mm]{geometry}

% переносы слов
\usepackage[english,russian]{babel}

% Таймс шрифт
\usepackage{tempora}

% полуторный межстрочный интервал
\usepackage{setspace}
\onehalfspacing

% отступ первой строки
\usepackage{indentfirst}
\setlength{\parindent}{1.25cm}

% структурный элемент
\newcommand{\hidedstructel}[1]{
    \clearpage
    \section*{#1}
}

\newcommand{\structel}[1]{
    \hidedstructel{#1}
    \addcontentsline{toc}{section}{#1} 
} 

\usepackage{titlesec}
\titleformat{\section}
    [display]               % форма
    {\filcenter\bfseries}   % формат всего заголовка
    {ПРИЛОЖЕНИЕ \thesection}                      % метка
    {}                      % отступ от метки
    {}                      % код перед телом

% раздел
\newcommand{\sect}[1]{
    \clearpage
    \setcounter{figure}{0}  % сбросить нумерацию внутри раздела
    \setcounter{table}{0}
    \setcounter{listing}{0}
    \subsection{#1}
}
\titleformat{\subsection}{\bfseries}{\thesubsection}{1em}{}
\titlespacing*{\subsection}
    {\parindent}    % отступ слева
    {}              % до
    {}              % после
\renewcommand{\thesubsection}{\arabic{subsection}}

\renewcommand{\thesection}{\Asbuk{section}}
\newcommand{\pril}[1]{
    \clearpage
    \section{#1}
}

% подраздел
\usepackage{placeins}
\newcommand{\subsect}[1]{
    \FloatBarrier
    \subsubsection{#1}
}
\titleformat{\subsubsection}{\bfseries}{\thesubsubsection}{1em}{}
\titlespacing*{\subsubsection}{\parindent}{}{}

% пункт
\newcommand{\foo}{
    \paragraph{hello world!}
}
\titleformat{\paragraph}[runin]{}{\theparagraph}{1em}{ }  % для отступа
\titlespacing*{\paragraph}{\parindent}{}{}

% подпункт
\newcommand{\subparag}{
    \subparagraph{}
}
\titleformat{\subparagraph}[runin]{}{\thesubparagraph}{1em}{ }
\titlespacing*{\subparagraph}{\parindent}{}{}

% содержание
\usepackage{etoc}

% перечисления
\usepackage{enumitem}
\setlist{
    topsep=0,                   % отступ сверху и снизу списка
    partopsep=0,                % то же самое
    leftmargin=0,               % отступ слева
    labelsep=0,                 % отступ метки
    align=left,                 % выравнивание метки
    listparindent=\parindent,   % отступ первой строки абзаца
    itemsep=0,                  % отступ между элементами
    parsep=0                    % отступ между абзацами и элементами
}
\setlist[itemize]{
    label=--~,  % в списках тире короткое, в тексте - длинное
    labelwidth=1.2em,
    itemindent=\parindent+\labelwidth
}
\setlist[enumerate]{
    label=\arabic*),
    labelwidth=1.4em,
    itemindent=\parindent+\labelwidth
}
 
% подписи
\usepackage[singlelinecheck=false]{caption}
\DeclareCaptionLabelSeparator{gost}{~---~}

% рисунок
\newcommand{\fig}[3][1]{
    \begin{figure}[h]
        \centering
        \includegraphics[width=#1\textwidth]{#2}
        \caption{#3}\label{#2}
    \end{figure}
}
\renewcommand{\thefigure}{\thesubsection.\arabic{figure}}
\DeclareCaptionLabelFormat{gostfigure}{Рисунок #2}
\captionsetup[figure]{justification=centering, labelformat=gostfigure, position=bottom}

% листинг
\usepackage[newfloat,cache=false]{minted}
\newenvironment{codepiece}[2]
{
    \VerbatimEnvironment
    \def\savedcaption{\caption{#2}\label{lst:#1}}
    \begin{listing}[h]
    \begin{minted}[
        frame=single,
        framesep=10pt,
        linenos,
        breaklines,
        fontsize=\small
    ]{text}
}{
    \end{minted}
    \savedcaption
    \end{listing}
}
\renewcommand{\thelisting}{\thesubsection.\arabic{listing}}

% код в таблице
\newenvironment{tabcode}[1]
{
    \VerbatimEnvironment
    \begin{minipage}[t]{#1\textwidth}
    \begin{minted}[breaklines, fontsize=\small]{text}
}{
    \end{minted}
    \end{minipage}
}

% математика
\usepackage{amsmath}

% графики
\usepackage{tikz, pgfplots}
\pgfplotsset{compat=newest}

% таблицы
\usepackage{array}
\usepackage{multirow}
\usepackage{longtable}
\usepackage{tabularx}
\renewcommand{\thetable}{\thesubsection.\arabic{table}}

\usepackage{adjustbox}
\usepackage{float}


\DeclareCaptionLabelFormat{gosttable}{Таблица #2}
\DeclareCaptionLabelFormat{custlisting}{Листинг #2}

\captionsetup[table]{justification=raggedright, labelformat=gosttable, position=top}
\captionsetup[listing]{justification=centering, labelformat=custlisting, position=bottom, skip=-6pt}
\captionsetup{labelsep=gost}


% для отладки
%\usepackage{showframe}
%\renewcommand\ShowFrameLinethickness{0.25pt}
%\renewcommand*\ShowFrameColor{\color{red}}


\begin{document}

\begin{titlepage}
    \begin{center}
        Министерство науки и высшего образования Российской Федерации\\
        Фереральное государственное бюджетное образовательное учреждение высшего образования\\
        \textsc{югорский государственный университет\\
        (югу)}
    \end{center}

    \vfill

    УДК: 004.89

    \newcommand{\signplace}{\underline{\hspace{40mm}}}
    \newcommand{\dateblank}{<<\underline{\hspace{10mm}}>> \underline{\hspace{30mm}} 2019 г.}
    \begin{tabular}{p{70mm}p{80mm}}
        & \textsc{утверждаю}\\
        Директор ИЦЭ,\\
        к.э.н
        &
        \signplace Ю.C.Родь\\
        & \dateblank
    \end{tabular}

    \vfill

    \begin{center}
        ПОЯСНИТЕЛЬНАЯ ЗАПИСКА
        \textsc{пояснительная записка\\
        к дипломному проекту бакалавра}\\
        \\
        \textsc{разработка серверной части интеллектуального информационного сервиса для образовательных организаций}\\
        (заключительный)
    \end{center}

    \vfill

    \begin{tabular}{p{50mm}p{80mm}}
        Научный руководитель,\\
        к.т.н
        &
        \signplace Е.И.Сафонов\\
        \vspace{5mm}
        Исполнитель,\\
        студент группы 1162б
        &
        \signplace И.Р.Панчишин\\
    \end{tabular}

    \vfill
    
    \begin{center}
        Ханты-Мансийск 2020
    \end{center}
\end{titlepage}

\setcounter{page}{2}


\hidedstructel{РЕФЕРАТ}

Пояснительная записка - с., - рис., - источн., - прил.

ЗАПИСКА, ПОЯСНЕНИЕ, ПРИМЕР ИСПОЛЬЗОВАНИЯ.

\begingroup

\parindent 0pt
\newlength{\pagewidth}\setlength{\pagewidth}{1.1em}

\newlength{\sectnum}\setlength{\sectnum}{8.4em}
\newlength{\ssectnum}\setlength{\ssectnum}{1em}
\newlength{\sssectnum}\setlength{\sssectnum}{2em}

\newlength{\sssectindent}\setlength{\sssectindent}{2em}

\newcommand*{\entrybody}{%
    \raggedright%
    \etocname\nobreak%
    \leaders\etoctoclineleaders\hfill%
    \rlap{\makebox[\pagewidth][r]{\etocpage}}%
    \vspace{0.56em}% хак для отступа
}

\etocsetstyle{section}
    {}
    {\leavevmode\etocifnumbered{\leftskip \sectnum}{}}
    {\normalsize\etocifnumbered%
        {\llap{\makebox[\sectnum][l]{ПРИЛОЖЕНИЕ \etocnumber}}%
            \parbox[t][][t]{\textwidth-\sectnum-\pagewidth}{\entrybody}}%
        {\parbox[t][][t]{\textwidth-\pagewidth}{\entrybody}}\par}
    {}
\etocsetstyle{subsection}
    {}
    {\leavevmode\leftskip \ssectnum}
    {\normalsize\llap{\makebox[\ssectnum][l]{\etocnumber}}%
        \parbox[t][][t]{\textwidth-\ssectnum-\pagewidth}{\entrybody}\par}
    {}
\etocsetstyle{subsubsection}
    {}
    {\leavevmode\setlength{\leftskip}{\sssectnum+\sssectindent}\relax}
    {\normalsize\llap{\makebox[\sssectnum][l]{\etocnumber}}%
        \parbox[t][][t]{\textwidth-\sssectnum-\pagewidth-\sssectindent}{\entrybody}\par}
    {}

\etocsettocstyle{\hidedstructel{СОДЕРЖАНИЕ}}{}

\tableofcontents

\endgroup


\structel{ТЕРМИНЫ И ОПРЕДЕЛЕНИЯ}

Обозначение

Заголовок

\structel{ВВЕДЕНИЕ ВВЕДЕНИЕ ВВЕДЕНИЕ ВВЕДЕНИЕ ВВЕДЕНИЕ ВВЕДЕНИЕ}

ГОСТ 7.32 не является строгим и не описывает все детали оформления документа.
На мой взгляд, это хорошо, так как использующие его организации получают
возможность уточнять стандарт по своему вкусу.

\sect{Оформление документа Оформление документа Оформление документа}

\subsect{Оформление содержания Оформление содержания Оформление содержания Оформление содержания}

Про содержание в стандарте написано в подразделах 5.4 и 6.13.

Каждая запись выравнена по левому краю, номер страницы --- по правому. Номер
страницы соединяется отточием. Подразделы начинаются после абзацного отступа,
равного двум знакам.

Про межстрочный интервал и расстояние между записями ничего не написано, но
указано, что "запись содержания оформляют как отдельный абзац", поэтому данные
параметры те же, что и в основной части документа.

Заголовок раздела при переносе продолжается от уровня своего начала, заголовок
приложения --- "от уровня записи обозначения этого приложения". Рекомендация
по приложению может быть не ясна. Чтобы точно понять, какой отступ у заголовка
приложения на новой строке, можно обратиться к ГОСТ 1.5-2001. По этому госту
оформляются другие госты, и в нем содежится та же рекомендация, с той же
формулировкой. Это значит, что можно посмотреть на отступ, например, в ГОСТ
7.32 и сделать так же.

\foo{} В содержание не включаются пункты.
\paragraph{hello} 

world
\subparag В стандарте указано, что как правило, они заголовков не имеют.

\sect{Обзор аналогов}

\end{document}
