\documentclass[a4paper,14pt]{extarticle}

% кодировка
\usepackage[utf8]{inputenc}
\usepackage[T2A]{fontenc}

% поля
\usepackage[left=30mm,right=15mm,top=20mm,bottom=20mm]{geometry}

% переносы слов
\usepackage[english,russian]{babel}

% Таймс шрифт
\usepackage{tempora}

% полуторный межстрочный интервал
\usepackage{setspace}
\onehalfspacing

% отступ первой строки
\usepackage{indentfirst}
\setlength{\parindent}{1.25cm}

% структурный элемент
\newcommand{\hidedstructel}[1]{
    \clearpage
    \section*{#1}
}

\newcommand{\structel}[1]{
    \hidedstructel{#1}
    \addcontentsline{toc}{section}{#1} 
} 

\usepackage{titlesec}
\titleformat{\section}
    [display]               % форма
    {\filcenter\bfseries}   % формат всего заголовка
    {ПРИЛОЖЕНИЕ \thesection}                      % метка
    {}                      % отступ от метки
    {}                      % код перед телом

% раздел
\newcommand{\sect}[1]{
    \clearpage
    \setcounter{figure}{0}  % сбросить нумерацию внутри раздела
    \setcounter{table}{0}
    \setcounter{listing}{0}
    \subsection{#1}

    \renewcommand{\theparagraph}{\thesubsection.\arabic{paragraph}}
}
\titleformat{\subsection}{\filright\bfseries}{}{}{\thesubsection\hspace{1em}}
\titlespacing*{\subsection}
    {\parindent}    % отступ слева
    {}              % до
    {}              % после
\renewcommand{\thesubsection}{\arabic{subsection}}

\renewcommand{\thesection}{\Asbuk{section}}
\newcommand{\pril}[1]{
    \clearpage
    \section{#1}
}

% подраздел
\usepackage{placeins}
\newcommand{\subsect}[1]{
    \FloatBarrier
    \subsubsection{#1}

    \renewcommand{\theparagraph}{\thesubsubsection.\arabic{paragraph}}
}
\titleformat{\subsubsection}{\filright\bfseries}{}{}{\thesubsubsection\hspace{1em}}
\titlespacing*{\subsubsection}{\parindent}{}{}

% пункт
\newcommand{\parag}{
    \paragraph{}
}
\titleformat{\paragraph}[runin]{}{\theparagraph}{1em}{ }  % для отступа
\titlespacing*{\paragraph}{\parindent}{}{}

% подпункт
\newcommand{\subparag}{
    \subparagraph{}
}
\titleformat{\subparagraph}[runin]{}{\thesubparagraph}{1em}{ }
\titlespacing*{\subparagraph}{\parindent}{}{}

% содержание
\usepackage{etoc}

\setcounter{tocdepth}{3}
\setcounter{secnumdepth}{5}

% перечисления
\usepackage{enumitem}
\setlist{
    topsep=0,                   % отступ сверху и снизу списка
    partopsep=0,                % то же самое
    leftmargin=0,               % отступ слева
    labelsep=0,                 % отступ метки
    align=left,                 % выравнивание метки
    listparindent=\parindent,   % отступ первой строки абзаца
    itemsep=0,                  % отступ между элементами
    parsep=0                    % отступ между абзацами и элементами
}
\setlist[itemize]{
    label=--~,  % в списках тире короткое, в тексте - длинное
    labelwidth=1.2em,
    itemindent=\parindent+\labelwidth
}
\setlist[enumerate]{
    label=\arabic*),
    labelwidth=1.4em,
    itemindent=\parindent+\labelwidth
}
%\makeatletter
%\AddEnumerateCounter{\asbuk}{\russian@alph}{щ}
%\makeatother
\AddEnumerateCounter*{\asbuk}{\c@asbuk}{7}
\newlist{asblist}{enumerate}{2}
\setlist[asblist]{
    label=\asbuk*),
    labelwidth=1.4em,
    itemindent=\parindent+\labelwidth
}

% подписи
\usepackage[singlelinecheck=false]{caption}
\DeclareCaptionLabelSeparator{gost}{~---~}

% иллюстрация
\newcommand{\fig}[3][1]{
    \begin{figure}[h]
        \centering
        \includegraphics[width=#1\textwidth]{#2}
        \caption{#3}\label{#2}
    \end{figure}
}
\renewcommand{\thefigure}{\thesubsection.\arabic{figure}}
\DeclareCaptionLabelFormat{gostfigure}{Рисунок #2}
\captionsetup[figure]{justification=centering, labelformat=gostfigure, position=bottom}
% font=singlespacing по умолчанию



% листинг
\usepackage[newfloat, cache=false]{minted}

\newcommand{\lst}[2]{
    \begin{listing}[h]
        \centering
        \caption{#2}\label{#1}
        \begin{minipage}[t]{.8\textwidth}
            \inputminted[
                fontsize=\small,
                frame=single,
                breaklines,
                linenos
            ]{text}{#1}
        \end{minipage}
    \end{listing}
}
\renewcommand{\theFancyVerbLine}{\rmfamily{\small \oldstylenums{\arabic{FancyVerbLine}}}} 
\DeclareCaptionLabelFormat{custlisting}{Листинг #2}
\captionsetup[listing]{justification=raggedright, labelformat=custlisting, position=top}
%skip=-6pt
\renewcommand{\thelisting}{\thesubsection.\arabic{listing}}

\newenvironment{codepiece}[2]
{
    \VerbatimEnvironment
    \begin{listing}[h]
        \centering
        \caption{#2}\label{lst:#1}
        \begin{minipage}[t]{.8\textwidth}
            \begin{minted}[
                fontsize=\small,
                frame=single,
                breaklines,
                linenos
            ]{text}
}{
            \end{minted}
        \end{minipage}
    \end{listing}
}

% код в таблице
\newenvironment{tabcode}[1]
{
    \VerbatimEnvironment
    \begin{minipage}[t]{#1\textwidth}
    \begin{minted}[fontsize=\small, breaklines]{text}
}{
    \end{minted}
    \end{minipage}
}

\usepackage{multirow}
\usepackage{longtable}
\usepackage{tabularx}
\renewcommand{\thetable}{\thesubsection.\arabic{table}}

% таблица
\newenvironment{tbl}[3]
{
    \begin{table}[h]
        \small
        \centering
        \caption{#2}\label{tbl:#1}
        \begin{tabular}{|#3|}
            \hline
}{
            \hline
        \end{tabular}
    \end{table}
}

\newenvironment{longtbl}[3]
{
    \small
    \begin{longtable}{|#3|}
        \caption{#2}\label{tbl:#1}\\
        \hline
}{
        \hline
    \end{longtable}
}

%\newcommand{\mr}[2]{\multirow[t]{#1}{=}{#2}}



% математика
\usepackage{amsmath}

% графики
\usepackage{tikz, pgfplots}
\pgfplotsset{compat=newest}

% таблицы
\usepackage{array}

\newcolumntype{M}[1]{>{\centering\arraybackslash}m{#1}}
\newcolumntype{N}[1]{>{\raggedright\arraybackslash}p{#1}}

\usepackage{xparse}
\NewExpandableDocumentCommand\thead{t< t> O{1} m m}{% 
    \IfBooleanTF{#1}{%
        \IfBooleanTF{#2}{%
            \multicolumn{#3}{|M{#4}|}{#5}%
        }{%
            \multicolumn{#3}{|M{#4}}{#5}%
        }
    }{%
        \IfBooleanTF{#2}{%
            \multicolumn{#3}{M{#4}|}{#5}%
        }{%
            \multicolumn{#3}{M{#4}}{#5}%
        }%
    }%
}




\usepackage{adjustbox}
\usepackage{float}


\DeclareCaptionLabelFormat{gosttable}{Таблица #2}

\captionsetup[table]{justification=raggedright, labelformat=gosttable, position=top}
\captionsetup{labelsep=gost}


% для отладки
%\usepackage{showframe}
%\renewcommand\ShowFrameLinethickness{0.25pt}
%\renewcommand*\ShowFrameColor{\color{red}}
% картинки
\usepackage{graphicx}


