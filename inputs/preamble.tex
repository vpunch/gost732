\documentclass[a4paper,14pt]{extarticle}

% кодировка
\usepackage[utf8]{inputenc}
\usepackage[T2A]{fontenc}

% поля
\usepackage[left=30mm,right=15mm,top=20mm,bottom=20mm]{geometry}

% переносы слов
\usepackage[english,russian]{babel}

% Таймс шрифт
\usepackage{tempora}

% полуторный межстрочный интервал
\usepackage{setspace}
\onehalfspacing

% отступ первой строки
\usepackage{indentfirst}
\setlength{\parindent}{1.25cm}

% структурный элемент
\newcommand{\hidedstructel}[1]{
    \clearpage
    \section*{#1}
}

\newcommand{\structel}[1]{
    \hidedstructel{#1}
    \addcontentsline{toc}{section}{#1} 
} 

\usepackage{titlesec}
\titleformat{\section}
    [display]               % форма
    {\filcenter\bfseries}   % формат всего заголовка
    {ПРИЛОЖЕНИЕ \thesection}                      % метка
    {}                      % отступ от метки
    {}                      % код перед телом

% раздел
\newcommand{\sect}[1]{
    \clearpage
    \setcounter{figure}{0}  % сбросить нумерацию внутри раздела
    \setcounter{table}{0}
    \setcounter{listing}{0}
    \subsection{#1}
}
\titleformat{\subsection}{\bfseries}{\thesubsection}{1em}{}
\titlespacing*{\subsection}
    {\parindent}    % отступ слева
    {}              % до
    {}              % после
\renewcommand{\thesubsection}{\arabic{subsection}}

\renewcommand{\thesection}{\Asbuk{section}}
\newcommand{\pril}[1]{
    \clearpage
    \section{#1}
}

% подраздел
\usepackage{placeins}
\newcommand{\subsect}[1]{
    \FloatBarrier
    \subsubsection{#1}
}
\titleformat{\subsubsection}{\bfseries}{\thesubsubsection}{1em}{}
\titlespacing*{\subsubsection}{\parindent}{}{}

% пункт
\newcommand{\foo}{
    \paragraph{hello world!}
}
\titleformat{\paragraph}[runin]{}{\theparagraph}{1em}{ }  % для отступа
\titlespacing*{\paragraph}{\parindent}{}{}

% подпункт
\newcommand{\subparag}{
    \subparagraph{}
}
\titleformat{\subparagraph}[runin]{}{\thesubparagraph}{1em}{ }
\titlespacing*{\subparagraph}{\parindent}{}{}

% содержание
\usepackage{etoc}

% перечисления
\usepackage{enumitem}
\setlist{
    topsep=0,                   % отступ сверху и снизу списка
    partopsep=0,                % то же самое
    leftmargin=0,               % отступ слева
    labelsep=0,                 % отступ метки
    align=left,                 % выравнивание метки
    listparindent=\parindent,   % отступ первой строки абзаца
    itemsep=0,                  % отступ между элементами
    parsep=0                    % отступ между абзацами и элементами
}
\setlist[itemize]{
    label=--~,  % в списках тире короткое, в тексте - длинное
    labelwidth=1.2em,
    itemindent=\parindent+\labelwidth
}
\setlist[enumerate]{
    label=\arabic*),
    labelwidth=1.4em,
    itemindent=\parindent+\labelwidth
}
 
% подписи
\usepackage[singlelinecheck=false]{caption}
\DeclareCaptionLabelSeparator{gost}{~---~}

% рисунок
\newcommand{\fig}[3][1]{
    \begin{figure}[h]
        \centering
        \includegraphics[width=#1\textwidth]{#2}
        \caption{#3}\label{#2}
    \end{figure}
}
\renewcommand{\thefigure}{\thesubsection.\arabic{figure}}
\DeclareCaptionLabelFormat{gostfigure}{Рисунок #2}
\captionsetup[figure]{justification=centering, labelformat=gostfigure, position=bottom}

% листинг
\usepackage[newfloat,cache=false]{minted}
\newenvironment{codepiece}[2]
{
    \VerbatimEnvironment
    \def\savedcaption{\caption{#2}\label{lst:#1}}
    \begin{listing}[h]
    \begin{minted}[
        frame=single,
        framesep=10pt,
        linenos,
        breaklines,
        fontsize=\small
    ]{text}
}{
    \end{minted}
    \savedcaption
    \end{listing}
}
\renewcommand{\thelisting}{\thesubsection.\arabic{listing}}

% код в таблице
\newenvironment{tabcode}[1]
{
    \VerbatimEnvironment
    \begin{minipage}[t]{#1\textwidth}
    \begin{minted}[breaklines, fontsize=\small]{text}
}{
    \end{minted}
    \end{minipage}
}

% математика
\usepackage{amsmath}

% графики
\usepackage{tikz, pgfplots}
\pgfplotsset{compat=newest}

% таблицы
\usepackage{array}
\usepackage{multirow}
\usepackage{longtable}
\usepackage{tabularx}
\renewcommand{\thetable}{\thesubsection.\arabic{table}}

\usepackage{adjustbox}
\usepackage{float}


\DeclareCaptionLabelFormat{gosttable}{Таблица #2}
\DeclareCaptionLabelFormat{custlisting}{Листинг #2}

\captionsetup[table]{justification=raggedright, labelformat=gosttable, position=top}
\captionsetup[listing]{justification=centering, labelformat=custlisting, position=bottom, skip=-6pt}
\captionsetup{labelsep=gost}


% для отладки
%\usepackage{showframe}
%\renewcommand\ShowFrameLinethickness{0.25pt}
%\renewcommand*\ShowFrameColor{\color{red}}
